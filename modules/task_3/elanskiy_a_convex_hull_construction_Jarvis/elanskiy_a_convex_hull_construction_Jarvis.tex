\documentclass{report}

\usepackage[warn]{mathtext}
\usepackage[T2A]{fontenc}
\usepackage[utf8]{luainputenc}
\usepackage[english, russian]{babel}
\usepackage[pdftex]{hyperref}
\usepackage{tempora}
\usepackage[12pt]{extsizes}
\usepackage{listings}
\usepackage{color}
\usepackage{geometry}
\usepackage{enumitem}
\usepackage{multirow}
\usepackage{graphicx}
\usepackage{indentfirst}
\usepackage{amsmath}

\geometry{a4paper,top=2cm,bottom=2cm,left=2.5cm,right=1.5cm}
\setlength{\parskip}{0.5cm}
\setlist{nolistsep, itemsep=0.3cm,parsep=0pt}

\usepackage{listings}
\lstset{language=C++,
        basicstyle=\footnotesize,
		keywordstyle=\color{blue}\ttfamily,
		stringstyle=\color{red}\ttfamily,
		commentstyle=\color{green}\ttfamily,
		morecomment=[l][\color{red}]{\#}, 
		tabsize=4,
		breaklines=true,
  		breakatwhitespace=true,
  		title=\lstname,       
}

\makeatletter
\renewcommand\@biblabel[1]{#1.\hfil}
\makeatother

\begin{document}

\begin{titlepage}

\begin{center}
Министерство науки и высшего образования Российской Федерации
\end{center}

\begin{center}
Федеральное государственное автономное образовательное учреждение высшего образования \\
Национальный исследовательский Нижегородский государственный университет им. Н.И. Лобачевского
\end{center}

\begin{center}
Институт информационных технологий, математики и механики
\end{center}

\vspace{4em}

\begin{center}
\textbf{\LargeОтчет по лабораторной работе} \\
\end{center}
\begin{center}
\textbf{\Large«Построение выпуклой оболочки – проход Джарвиса.»} \\
\end{center}

\vspace{4em}

\newbox{\lbox}
\savebox{\lbox}{\hbox{text}}
\newlength{\maxl}
\setlength{\maxl}{\wd\lbox}
\hfill\parbox{7cm}{
\hspace*{5cm}\hspace*{-5cm}\textbf{Выполнила:} \\ студент группы 381908-3 \\ Еланский А. А.\\
\\
\hspace*{5cm}\hspace*{-5cm}\textbf{Проверил:}\\ доцент кафедры МОСТ, \\ кандидат технических наук \\ Сысоев А. В.\\
}
\vspace{\fill}

\begin{center} Нижний Новгород \\ 2021 \end{center}

\end{titlepage}

\setcounter{page}{2}

% Содержание
\tableofcontents
\newpage

% Введение
\section*{Введение}
\addcontentsline{toc}{section}{Введение}
\par Алгоритм обхода Джарвиса определяет последовательность элементов множества, образующих выпуклую оболочку для этого множества. Метод можно представить как обтягивание верёвкой множества вбитых в доску гвоздей. Алгоритм работает за время O(nh), где n — общее число точек на плоскости, h — число точек в выпуклой оболочке.
\newpage

% Постановка задачи
\section*{Постановка задачи}
\addcontentsline{toc}{section}{Постановка задачи}
\par В данном лабораторной работе требуется реализовать последовательный и параллельный алгоритмы вычисления Построение выпуклой оболочки – проход Джарвиса. Изпользуя автоматические тесты Google Test, провести вычислительный эксперименты для сравнения времени работы алгоритмов, 
\par Параллельный алгоритм должен быть реализован при помощи технологии MPI.
\newpage

% Описание алгоритма
\section*{Описание алгоритма}
\addcontentsline{toc}{section}{Описание алгоритма}
Пусть на плоскости задано конечное множество точек $A=\{\,a_{1},a_{2},\ldots ,a_{n}\,\}$. Оболочкой этого множества называется любая замкнутая линия H без самопересечений, такая, что все точки из множества A лежат внутри этой кривой. Если кривая H является выпуклой, то соответствующая оболочка также называется выпуклой. Минимальной выпуклой оболочкой (МВО) называется выпуклая оболочка минимальной длины (минимального периметра).

\par Главной особенностью МВО множества точек A является то, что эта оболочка представляет собой выпуклый многоугольник, вершинами которого являются некоторые точки из A. Поэтому задача поиска МВО в конечном итоге сводится к отбору и упорядочиванию нужных точек из A. Упорядочивание является необходимым по той причине, что выходом алгоритма должен быть многоугольник, т.е. последовательность вершин. 

\par Алгоитм Джарвиса состоит из 2 шагов. В первом шаге ищется любая точка в A, гарантированно входящая в МВО. Она будет входить в МВО, значит, берётся точка с наименьшей x-координатой (самая левая точка в A). Эту точку (стартовая) перемещаем в начало списка, вся дальнейшая работа будет производиться с оставшимися точками. На втором шаге алгоритма строится МВО. По такому принципу: делаем стартовую вершину текущей, ищем самую правую точку в A относительно текущей вершины, делаем ее текущей и повторяем. Процесс завершается, когда текущей вновь окажется стартовая вершина. 
\newpage

% Описание схемы распараллеливания
\section*{Описание схемы распараллеливания}
\addcontentsline{toc}{section}{Описание схемы распараллеливания}
\par Количество точек, которые обработывает каждый из процессов, вычисляется по формуле $N_{\text{local}} = \frac{\text{общее количество точек}}{\text{количество процессов}}$. Если при делении образуется остаток, то эти оставшиеся точки отдаются на обработку последнему процессу. Каждый процесс вычисляет значение МВО для своих точек. Потом значения МВО каждого процесса скадываются в один вектор, в котором мы находим МВО. В итоге, мы получаем МВО для всех точек.
\newpage

% Описание программной реализации
\section*{Описание программной реализации}
\addcontentsline{toc}{section}{Описание программной реализации}
Программа состоит из заголовочного файла convex\_hull\_construction\_Jarvis.h, в котором находятся объвления функций, и двух исполняющих файлов convex\_hull\_construction\_Jarvis.cpp, в котором находятся реализация функций, и main.cpp, в котором находятся тесты для проверки значений. 
\par В заголовочном файле объявляются функции для генерации точек, определение местоположения точки, последовательного и параллельного вычисления МВО.

\par Функция для генерации точек:
\begin{lstlisting}
std::vector<std::pair<int, int>> gen_dots(int vectorSize);
\end{lstlisting}
Во входной параметр принимается количество точек, которые мы хотим задать. А возвращает функция вектор точек типа пара, с целочисленными координатами точек

\par Функция для определение местоположения точки:
\begin{lstlisting}
int rotation(const std::pair<int, int>& firstVertex, const std::pair<int, int>& secondVertex,
    const std::pair<int, int>& thirdVertex);
\end{lstlisting}
Входные параметры данной функции являются точки $\{\,a_{предпоследняя точка в векторе точек},a_{крайняя точка},a_{n}\,\}$. На выходе мы получаем число, с помощью которого определяем местоположение следующей точки относительно текущей.

\par Функция для последовательного вычисления МВО:
\begin{lstlisting}
std::vector<std::pair<int, int>> jarvis_seq(const std::vector<std::pair<int, int>>& vectorOfVertex);
\end{lstlisting}
Во входной параметр принимается вектор точек, из которых мы хотим найти МВО. На выходе мы получаем вектор точек, которые образуют МВО.

\par Функция для параллельного вычисления МВО:
\begin{lstlisting}
std::vector<std::pair<int, int>> jarvis_par(const std::vector<std::pair<int, int>>& vectorOfVertex, int vectorSize);
\end{lstlisting}
Во входной параметр принимается вектор точек, из которых мы хотим найти МВО и количество точек в векторе. На выходе мы получаем вектор точек, которые образуют МВО.
\newpage

% Подтверждение корректности
\section*{Подтверждение корректности}
\addcontentsline{toc}{section}{Подтверждение корректности}
Для подтверждения корректности работы данной программы с помощью фрэймфорка Google Test, были разработаны 7 тестов. В каждом из тестов вычисляются МВО через последовательный и  алгоритмы. Из случайно сгенерированных точек оба алгоритма находится точки МВО и сравниваются между собой. 

\par Алгоритмы тестируются по следующим количеству точек: 31, 32, 100, 123, 222, 256 и 5000. Успешное прохождение тестов по этим значениям подтверждает корректность работы алгоритмов.
\newpage

% Результаты экспериментов
\section*{Результаты экспериментов}
\addcontentsline{toc}{section}{Результаты экспериментов}
Вычислительные эксперименты для оценки эффективности работы параллельного алгоритма проводились на ПК со следующими характеристиками:
\begin{itemize}
\item Процессор: AMD Ryzen 7 2700x, 3.7 ГГц, количество ядер: 8;
\item Оперативная память: 16 ГБ (DDR4);
\item Операционная система: Windows 10 Home.
\end{itemize}

\par Эксперименты проводились на различном числе процессов и на количестве точек 5000.

\par Результаты экспериментов представлены в Таблице 1.

\begin{table}[!h]
\caption{Результат вычислительного эксперимента}
\centering
\begin{tabular}{| p{2cm} | p{3cm} | p{4cm} | p{2cm} |}
\hline
Количество процессов & Время работы параллельного алгоритма (в секундах) & Время работы последовательного алгоритма (в секундах)   \\[5pt]
\hline
2        & 0.0093039        & 0.0175132            \\
3        & 0.006356         & 0.0195783            \\
4        & 0.0044646        & 0.0153115            \\
5        & 0.0035047        & 0.0155628            \\
6        & 0.0033614        & 0.0152676            \\
7        & 0.002986         & 0.0231004            \\
8        & 0.0033797        & 0.0171767            \\

\hline
\end{tabular}
\end{table}

\newpage

% Выводы из результатов экспериментов
\section*{Выводы из результатов экспериментов}
\addcontentsline{toc}{section}{Выводы из результатов экспериментов}
Из данных, полученных в результате экспериментов (см. Таблицу 1), можно сделать вывод, что параллельный алгоритм работает быстрее, чем последовательный. Можно заметить, что с увеличением числа процессов сокращается время работы параллельного алгоритма. Однако, начиная с некоторого количества процессов, время выполнения алгоритма снижается не очень сильно, по сравнению с предыдущем количеством процессов. Это связано с тем, что, чем больше процессом, тем больше времени уходит на распределение данных. 

\newpage

% Заключение
\section*{Заключение}
\addcontentsline{toc}{section}{Заключение}
Подводя итоги, можно сказать, что в данной лабораторной работе были разработаны последовательный и параллельный алгоритмы Построение выпуклой оболочки алгоритмом Джарвиса. Проведенные тесты показали корректность реализованной программы, а проведенные эксперименты доказали эффективность параллельного алгоритма по сравнению с последовательным.
\newpage

% Литература
\section*{Литература}
\addcontentsline{toc}{section}{Литература}
\begin{enumerate}
\item Гергель В.П.,Сысоев А.В. Кафедра МОСТ: Презентация «Лекция 02 Коллективные и парные взаимодействия»
\item Habr - Электронный ресурс. URL: https://habr.com/ru/post/144921/
\item Википедия - Электронный ресурс. URL: https://ru.wikipedia.org/wiki/%D0%90%D0%BB%D0%B3%D0%BE%D1%80%D0%B8%D1%82%D0%BC_%D0%94%D0%B6%D0%B0%D1%80%D0%B2%D0%B8%D1%81%D0%B0
\end{enumerate} 
\newpage

% Приложение
\section*{Приложение}
\addcontentsline{toc}{section}{Приложение}

convex\_hull\_construction\_Jarvis.h
\begin{lstlisting}
/// Copyright 2021 Elanskiy Akexandr
#include <cstdlib>
#include <ctime>
#include <gtest-mpi-listener.hpp>
#include <gtest/gtest.h>
#include <mpi.h>
#include <numeric>
#include <random>
#include <utility>
#include <vector>

#ifndef MODULES_TASK_3_ELANSKIY_A_CONVEX_HULL_CONSTRUCTION_JARVIS_CONVEX_HULL_CONSTRUCTION_JARVIS_H_
#define MODULES_TASK_3_ELANSKIY_A_CONVEX_HULL_CONSTRUCTION_JARVIS_CONVEX_HULL_CONSTRUCTION_JARVIS_H_

std::vector<std::pair<int, int>> gen_dots(int vectorSize);
int rotation(const std::pair<int, int>& firstVertex, const std::pair<int, int>& secondVertex,
    const std::pair<int, int>& thirdVertex);
std::vector<std::pair<int, int>> jarvis_seq(const std::vector<std::pair<int, int>>& vectorOfVertex);
std::vector<std::pair<int, int>> jarvis_par(const std::vector<std::pair<int, int>>& vectorOfVertex, int vectorSize);

#endif // MODULES_TASK_3_ELANSKIY_A_CONVEX_HULL_CONSTRUCTION_JARVIS_CONVEX_HULL_CONSTRUCTION_JARVIS_H_


\end{lstlisting}
convex\_hull\_construction\_Jarvis.cpp
\begin{lstlisting}
// Copyright 2021 Elanskiy Akexandr
#include "../../../modules/task_3/elanskiy_a_convex_hull_construction_Jarvis/convex_hull_construction_Jarvis.h"

std::vector<std::pair<int, int>> gen_dots(int vectorSize) {
    std::mt19937 rand_r(time(0));
    std::vector<std::pair<int, int>> vec(vectorSize);
    for (int i = 0; i < vectorSize; i++) {
        vec[i].first = rand_r() % 10000;
        vec[i].second = rand_r() % 10000;
    }
    return vec;
}

int rotation(const std::pair<int, int>& dot1, const std::pair<int, int>& dot2, const std::pair<int, int>& dot3) {
    return (dot2.first - dot1.first) * (dot3.second - dot2.second)
        - (dot2.second - dot1.second) * (dot3.first - dot2.first);
}

std::vector<std::pair<int, int>> jarvis_seq(const std::vector<std::pair<int, int>>& dots) {
    std::vector<int> _indxs(dots.size());
    int vec_size = dots.size();
    int tmp;
    for (int i = 0; i < vec_size; i++) {
        _indxs[i] = i;
    }
    for (int i = 1; i < vec_size; ++i) {
        if (dots[_indxs[i]].first < dots[_indxs[0]].first) {
            tmp = _indxs[i];
            _indxs[i] = _indxs[0];
            _indxs[0] = tmp;
        }
    }

    std::vector<int> res_indx = { _indxs[0] };
    _indxs.erase(_indxs.begin());
    _indxs.push_back(res_indx[0]);

    int _indxsSize;
    while (true) {
        int right = 0;
        _indxsSize = _indxs.size();
        for (int i = 1; i < _indxsSize; i++) {
            if (rotation(dots[res_indx[res_indx.size() - 1]],
                    dots[_indxs[right]],
                    dots[_indxs[i]])
                < 0) {
                right = i;
            }
        }
        if (_indxs[right] == res_indx[0]) {
            break;
        } else {
            res_indx.push_back(_indxs[right]);
            _indxs.erase(_indxs.begin() + right);
        }
    }

    std::vector<std::pair<int, int>> res(res_indx.size());
    int res_indxSize = res_indx.size();
    for (int i = 0; i < res_indxSize; i++) {
        res[i] = dots[res_indx[i]];
    }

    return res;
}

void Build_mpi_type(int* index_i, int* index_j, MPI_Datatype* mytype) {
    int array_of_blocklengths[2] = { 1, 1 };
    MPI_Datatype array_of_types[2] = { MPI_INT, MPI_INT };
    MPI_Aint i_addr, j_addr;
    MPI_Aint array_of_displacements[2] = { 0 };
    MPI_Get_address(index_i, &i_addr);
    MPI_Get_address(index_j, &j_addr);
    array_of_displacements[1] = j_addr - i_addr;
    MPI_Type_create_struct(2, array_of_blocklengths, array_of_displacements, array_of_types, mytype);
    MPI_Type_commit(mytype);
}

std::vector<std::pair<int, int>> jarvis_par(const std::vector<std::pair<int, int>>& dots, int size) {
    int num_of_proc;
    int proc_num;
    MPI_Comm_size(MPI_COMM_WORLD, &num_of_proc);
    MPI_Comm_rank(MPI_COMM_WORLD, &proc_num);

    std::vector<int> num_of_dots(num_of_proc);
    std::vector<int> displs(num_of_proc);
    std::vector<int> displs1(num_of_proc);
    std::vector<int> size_of_parts(num_of_proc);
    int sum = 0;

    int* index_i = NULL;
    int* index_j = NULL;
    std::pair<int, int> bla;
    index_i = &(bla.first);
    index_j = &(bla.second);
    MPI_Datatype mytype;
    Build_mpi_type(index_i, index_j, &mytype);

    for (int i = 0; i < num_of_proc; i++) {
        num_of_dots[i] = size / num_of_proc;
        displs[i] = sum;
        sum += num_of_dots[i];
    }
    if (size % num_of_proc > 0) {
        num_of_dots[num_of_proc - 1] += size % num_of_proc;
    }
    std::vector<std::pair<int, int>> vec_part(num_of_dots[proc_num]);
    std::vector<std::pair<int, int>> part_res;
    std::vector<std::pair<int, int>> res;
    int res_size = 0;

    MPI_Scatterv(dots.data(), num_of_dots.data(), displs.data(), mytype, vec_part.data(),
        num_of_dots[proc_num], mytype, 0, MPI_COMM_WORLD);

    part_res = jarvis_seq(vec_part);

    int part_size = part_res.size();
    MPI_Gather(&part_size, 1, MPI_INT, size_of_parts.data(), 1, MPI_INT, 0, MPI_COMM_WORLD);
    MPI_Bcast(size_of_parts.data(), size_of_parts.size(), MPI_INT, 0, MPI_COMM_WORLD);
    MPI_Reduce(&size_of_parts[proc_num], &res_size, 1, MPI_INT, MPI_SUM, 0, MPI_COMM_WORLD);
    MPI_Bcast(&res_size, 1, MPI_INT, 0, MPI_COMM_WORLD);

    sum = 0;
    for (int i = 0; i < num_of_proc; i++) {
        displs1[i] = sum;
        sum += size_of_parts[i];
    }
    std::vector<std::pair<int, int>> tmp_res(res_size);
    MPI_Gatherv(part_res.data(), size_of_parts[proc_num], mytype, tmp_res.data(),
        size_of_parts.data(), displs1.data(), mytype, 0, MPI_COMM_WORLD);
    res = jarvis_seq(tmp_res);
    return res;
}
\end{lstlisting}
main.cpp
\begin{lstlisting}
// Copyright 2021 Elanskiy Akexandr
#include "./convex_hull_construction_Jarvis.h"

TEST(Parallel_Operations_MPI, Test_Size_31) {
    int rank;
    MPI_Comm_rank(MPI_COMM_WORLD, &rank);
    std::vector<std::pair<int, int>> dots;
    int size = 31;
    dots = gen_dots(size);
    std::vector<std::pair<int, int>> par_res;
    std::vector<std::pair<int, int>> seq_res;
    par_res = jarvis_par(dots, size);
    if (rank == 0) {
        seq_res = jarvis_seq(dots);
        ASSERT_EQ(par_res, seq_res);
    }
}

TEST(Parallel_Operations_MPI, Test_Size_32) {
    int rank;
    MPI_Comm_rank(MPI_COMM_WORLD, &rank);
    std::vector<std::pair<int, int>> dots;
    int size = 31;
    dots = gen_dots(size);
    std::vector<std::pair<int, int>> par_res;
    std::vector<std::pair<int, int>> seq_res;
    par_res = jarvis_par(dots, size);
    if (rank == 0) {
        seq_res = jarvis_seq(dots);
        ASSERT_EQ(par_res, seq_res);
    }
}

TEST(Parallel_Operations_MPI, Test_Size_100) {
    int rank;
    MPI_Comm_rank(MPI_COMM_WORLD, &rank);
    std::vector<std::pair<int, int>> dots;
    int size = 100;
    dots = gen_dots(size);
    std::vector<std::pair<int, int>> par_res;
    std::vector<std::pair<int, int>> seq_res;
    par_res = jarvis_par(dots, size);
    if (rank == 0) {
        seq_res = jarvis_seq(dots);
        ASSERT_EQ(par_res, seq_res);
    }
}

TEST(Parallel_Operations_MPI, Test_Size_123) {
    int rank;
    MPI_Comm_rank(MPI_COMM_WORLD, &rank);
    std::vector<std::pair<int, int>> dots;
    int size = 123;
    dots = gen_dots(size);
    std::vector<std::pair<int, int>> par_res;
    std::vector<std::pair<int, int>> seq_res;
    par_res = jarvis_par(dots, size);
    if (rank == 0) {
        seq_res = jarvis_seq(dots);
        ASSERT_EQ(par_res, seq_res);
    }
}

TEST(Parallel_Operations_MPI, Test_Size_222) {
    int rank;
    MPI_Comm_rank(MPI_COMM_WORLD, &rank);
    std::vector<std::pair<int, int>> dots;
    int size = 222;
    dots = gen_dots(size);
    std::vector<std::pair<int, int>> par_res;
    std::vector<std::pair<int, int>> seq_res;
    par_res = jarvis_par(dots, size);
    if (rank == 0) {
        seq_res = jarvis_seq(dots);
        ASSERT_EQ(par_res, seq_res);
    }
}

TEST(Parallel_Operations_MPI, Test_Size_256) {
    int rank;
    MPI_Comm_rank(MPI_COMM_WORLD, &rank);
    std::vector<std::pair<int, int>> dots;
    int size = 256;
    dots = gen_dots(size);
    std::vector<std::pair<int, int>> par_res;
    std::vector<std::pair<int, int>> seq_res;
    par_res = jarvis_par(dots, size);
    if (rank == 0) {
        seq_res = jarvis_seq(dots);
        ASSERT_EQ(par_res, seq_res);
    }
}

TEST(Parallel_Operations_MPI, Test_Size_5000) {
    int rank;
    MPI_Comm_rank(MPI_COMM_WORLD, &rank);
    std::vector<std::pair<int, int>> dots;
    int size = 10000;
    dots = gen_dots(size);
    std::vector<std::pair<int, int>> par_res;
    double start_of_par;
    double end_of_par;
    double start_of_seq;
    double end_of_seq;
    if (rank == 0) {
        start_of_par = MPI_Wtime();
    }
    par_res = jarvis_par(dots, size);
    if (rank == 0) {
        end_of_par = MPI_Wtime();
        start_of_seq = MPI_Wtime();
        std::vector<std::pair<int, int>> seq_res = jarvis_seq(dots);
        end_of_seq = MPI_Wtime();
        std::cout << std::endl;
        std::cout << end_of_par - start_of_par << " par" << std::endl;
        std::cout << end_of_seq - start_of_seq << " seq" << std::endl;
        ASSERT_EQ(seq_res, seq_res);
    }
}

int main(int argc, char** argv) {
    ::testing::InitGoogleTest(&argc, argv);
    MPI_Init(&argc, &argv);

    ::testing::AddGlobalTestEnvironment(new GTestMPIListener::MPIEnvironment);
    ::testing::TestEventListeners& listeners = ::testing::UnitTest::GetInstance()->listeners();

    listeners.Release(listeners.default_result_printer());
    listeners.Release(listeners.default_xml_generator());

    listeners.Append(new GTestMPIListener::MPIMinimalistPrinter);
    return RUN_ALL_TESTS();
}
\end{lstlisting}

\end{document}